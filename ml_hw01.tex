%-----------------------Homework 01------------------------------------
%-------------------Arman Shokrollahi---------------------------------
%-------------------Modified by Jonghyun Choi------------------------

\documentclass[a4 paper]{article}
% Set target color model to RGB
\usepackage[inner=1.5cm,outer=1.5cm,top=2.5cm,bottom=2.5cm]{geometry}
\usepackage{setspace}
\usepackage[rgb]{xcolor}
\usepackage{verbatim}
\usepackage{amsgen,amsmath,amstext,amsbsy,amsopn,tikz,amssymb,tkz-linknodes}
\usepackage{fancyhdr}
\usepackage[colorlinks=true, urlcolor=blue,  linkcolor=blue, citecolor=blue]{hyperref}
\usepackage[colorinlistoftodos]{todonotes}
\usepackage{rotating}
\usepackage{enumitem}
%\usetikzlibrary{through,backgrounds}
\hypersetup{%
pdfauthor={Jonghyun Choi},%
pdftitle={Homework},%
pdfkeywords={Tikz,latex,bootstrap,uncertaintes},%
pdfcreator={PDFLaTeX},%
pdfproducer={PDFLaTeX},%
}
\usepackage[francais]{babel}
\usepackage{booktabs}
\newcommand{\ra}[1]{\renewcommand{\arraystretch}{#1}}

      \newtheorem{thm}{Theorem}[section]
      \newtheorem{prop}[thm]{Proposition}
      \newtheorem{lem}[thm]{Lemma}
      \newtheorem{cor}[thm]{Corollary}
      \newtheorem{defn}[thm]{Definition}
      \newtheorem{rem}[thm]{Remark}
      %\numberwithin{equation}{section}

\newcommand{\homework}[5]{
   \pagestyle{myheadings}
   \thispagestyle{plain}
   \newpage
   \setcounter{page}{1}
   \noindent
   \begin{center}
   \framebox{
      \vbox{\vspace{2mm}
    \hbox to 6.50in { {\bf EC4213/ET5402/CT5303:~Machine learning and deep learning \hfill Fall 2018} }
       \vspace{4mm}
       \hbox to 6.28in { {\Large \hfill #1 \hfill}}
       \vspace{1mm}
       \hbox to 6.28in { {\hfill (#2)  \hfill} }
       \vspace{3mm}
       \hbox to 6.28in { {\it Instructor: #3 \hfill Student name (ID\#): \hspace{8em}} }
       %\hbox to 6.28in { {\it TA: #4  \hfill #6}}
      \vspace{2mm}}
   }
   \end{center}
   \markboth{#5 -- #1}{#5 -- #1}
   \vspace*{4mm}
}

\newcommand{\bbF}{\mathbb{F}}
\newcommand{\bbX}{\mathbb{X}}
\newcommand{\bI}{\mathbf{I}}
\newcommand{\bX}{\mathbf{X}}
\newcommand{\bY}{\mathbf{Y}}
\newcommand{\bepsilon}{\boldsymbol{\epsilon}}
\newcommand{\balpha}{\boldsymbol{\alpha}}
\newcommand{\bbeta}{\boldsymbol{\beta}}
\newcommand{\0}{\mathbf{0}}
\DeclareMathOperator*{\argmin}{argmin}  
\DeclareMathOperator*{\argmax}{argmax}  

\usepackage{titlesec}
\titlespacing*{\section}
{1.5em}{1.5em}{1em}
\titlespacing*{\subsection}
{1.5em}{1em}{1em}

\renewcommand{\baselinestretch}{1.2}


\begin{document}
\homework{Homework 01}{due October 8, 2018}{Jonghyun Choi}{}{}

Please specify your name and your student ID in the top heading. Hand in as a fully answered version to github classroom.

%============================================================ 
\section*{Problem 1: Maximum Likelihood Estimation (30 pt)}
Maximum Likelihood Estimation (MLE) is a method of estimating the parameters of a statistical model given observations, by finding the parameters that maximize the likelihood of the observations. Concretely, given observations $y_1, y_2,..., y_n$ distributed according to $p_\theta (y_1, y_2,..., y_n)$ (here $p_\theta$ can be a probability mass function for discrete observations or a density for continuous observations), the likelihood function is defined as $L(\theta) = p_\theta (y_1, y_2,..., y_n)$ and the MLE is
\begin{align*}
    \hat{\theta}_{MLE} = \argmax_\theta L(\theta)
\end{align*}

We often make the assumption that the observations are independent and identically distributed or iid, in which case $p_\theta (y_1, y_2,..., y_n) = f_\theta (y_1)· f_\theta (y_2)····· f_\theta (y_n)$.

\begin{enumerate}[label=(\alph*)]
    \item The instructor recommends maximizing the log-likelihood $\ell (\theta) = \log{L(\theta)}$ instead of maximizing $L(\theta)$. Why does this yield the same solution $\hat{\theta}_{MLE}$? 
    
    \item Why is it easier to solve the optimization problem for $\ell (\theta)$ in the iid case? 
        
    \item The Poisson distribution is $f_\lambda(y) = \frac{\lambda^y e^{-\lambda}}{y!}$. Let $Y_1,Y_2,...,Y_n$ be a set of independent and identically distributed random variables with Poisson distribution with parameter $\lambda$. Find the joint distribution of $Y_1,Y_2,...,Y_n.$ 
    
    \item Find the maximum likelihood estimator of $\lambda$ as a function of observations $y_1, y_2,..., y_n$.
\end{enumerate}

\section*{Problem 2: The accuracy of learning decision boundaries (40 pt)}

This problem exercises your basic probability in the context of understanding why lots of training data helps to improve the accuracy of learning things.

For each $\theta \in (\frac{1}{3},\frac{2}{3})$, define $f_\theta : [0,1] \rightarrow \{0,1\}$, such that
\begin{align*}
    f_\theta(x) =
    \begin{cases}
    1 & \text{if } x>\theta \\
    0 & \text{otherwise.}
    \end{cases}
\end{align*}

\textit{(Hint: draw the plot of $f_\theta (x)$ w.r.t. $x$ may help you to solve the problems.)}

We draw samples $X_1,X_2,...,X_n$ uniformly at random and i.i.d. from the interval $[0,1]$. Our goal is to learn an estimate for $\theta$ from $n$ random samples $(X_1, f_\theta (X_1)),(X_2, f_\theta (X_2)),...,(X_n, f_\theta (X_n)).$ 

\begin{itemize}
    \item Let $T_{min} = \max (\{\frac{1}{3} \cup \{X_i | f_\theta(X_i) = 0 \}).$ We know that the true $\theta$ must be larger than $T_{min}.$

    \item Let $T_{max} = \min(\{\frac{2}{3} \cup \{X_i | f_\theta(X_i) = 1. \}).$ We know that the true $\theta$ must be smaller than $T_{max}$.
\end{itemize}

The gap between $T_{min}$ and $T_{max}$ represents the uncertainty we will have about the true $\theta$ given the training data that we have received.

\begin{enumerate}[label=(\alph*)]
    \item What is the probability that $T_{max}$ - $\theta > \alpha$ as a function of $\alpha$? 
    
    \item What is the probability that $\theta$ - $T_{min} > \alpha$ as a function of $\alpha$?
    
    \item Suppose that you would like the estimator $\hat{\theta} = (T_{max} + T_{min})$ for $\theta$ that is $\alpha$-close (defined as $|\hat{\theta} - \theta| < \alpha$, where $\hat{\theta}$ is the estimation and $\theta$ is the true value) with probability at least $1 - \delta$. Both $\alpha$ and $\delta$ are some small positive numbers. Please bound or estimate how big of an $n$ do you need? (\textbf{Note}: You do not need to find the optimal lowest sample complexity $n$, an approximation using results of question (a) is fine.)
\end{enumerate}


\section*{Problem 3: Geometry of Ridge Regression (20 pt)}
You recently learned ridge regression and how it differs from ordinary least squares. In this question we will explore how ridge regression is related to solving a constrained least squares problem in terms of their parameters and solutions.

Recall that ridge regression is given by the unconstrained optimization problem 
\begin{equation}
    \min_{\mathbf{w}} \|\mathbf{y-Xw}\|^2_2 + \lambda \|\mathbf{w}\|^2_2.
\label{eq:rr}
\end{equation}
    
\begin{enumerate}[label=(\alph*)]
    \item Derive that the solution to ridge regression~(\ref{eq:rr}) is given by $\mathbf{\hat{w_r}} = (\mathbf{X^\top X} + \lambda \mathbf{I})^{-1} \mathbf{X^\top y}$. 
    
    \item In the solution to the above problem, What happens when $\lambda \rightarrow \infty$? (Hint: It is for this reason that sometimes regularization is referred to as "shrinkage".)
            
\end{enumerate}

\end{document}
%%------------ Arman Shokrollahi--------------%%
%%------------ Modified by Jonghyun Choi--------------%%
